\documentclass[12pt]{article}

\usepackage[margin=1in]{geometry}
\usepackage{setspace}
\onehalfspacing
\usepackage{graphicx}
\graphicspath{report_images/}
\usepackage{appendix}
\usepackage{listings}
\usepackage{float}
\usepackage{multirow}
\usepackage{amsthm}
% The next three lines make the table and figure numbers also include section number
\usepackage{chngcntr}
\counterwithin{table}{section}
\counterwithin{figure}{section}
% Needed to make titling page without a page number
\usepackage{titling}

% This section required for the code block stuff below
\usepackage{color}
\definecolor{codegreen}{rgb}{0,0.6,0}
\definecolor{codegray}{rgb}{0.5,0.5,0.5}
\definecolor{codepurple}{rgb}{0.58,0,0.82}
\definecolor{backcolour}{rgb}{0.95,0.95,0.92}

% Makes code blocks have shaded backgrounds and line numbers
\lstdefinestyle{mystyle}{
    backgroundcolor=\color{backcolour},   
    commentstyle=\color{codegreen},
    keywordstyle=\color{magenta},
    numberstyle=\tiny\color{codegray},
    stringstyle=\color{codepurple},
    basicstyle=\footnotesize,
    breakatwhitespace=false,         
    breaklines=true,                 
    captionpos=b,                    
    keepspaces=true,                 
    numbers=left,                    
    numbersep=5pt,                  
    showspaces=false,                
    showstringspaces=false,
    showtabs=false,                  
    tabsize=2
}
 
\lstset{style=mystyle}

% DOCUMENT INFORMATION =================================================
\font\titleFont=cmr12 at 11pt
\title {{\titleFont ECEN 410:  Linear Control System \\ North Carolina Agricultural and Technical State University \\ Department of Electrical and Computer Engineering \\ Dr. Christopher Horne}} % Declare Title
\author{\titleFont  Chris Cannon} % Declare authors
\date{\titleFont November 29, 2018}
% ======================================================================

\begin{document}

\begin{titlingpage}
\maketitle
\begin{center}
	Exam 2 Research Paper
\end{center}
\end{titlingpage}

\tableofcontents

\pagebreak

\section{Introduction}

This paper provides analysis of a control system of the azimuth position of an antenna. This is a critical controls problem in today's world, where immediate access to information is considered an absolute necessity. Emergency response personnel, military entities, other government organizations, and even private citizens depend upon immediate access to information in critical situations every day. This analysis will focus on the response of the system to various inputs, by determining the transfer function of each subsystem as well as the system as a whole.

\begin{figure}[H]
\begin{center}
	\includegraphics[width=0.8\textwidth]{./img/SystemLayout.png}
	\caption{\label{fig:SysLayout}Control system layout}
\end{center}
\end{figure}

\begin{figure}[H]
\begin{center}
	\includegraphics[width=0.8\textwidth]{./img/SystemSchematic.png}
	\caption{\label{fig:SysSchematic}Control system schematic}
\end{center}
\end{figure}

\begin{figure}[H]
\begin{center}
	\includegraphics[width=0.8\textwidth]{./img/SystemBlockDiagram.png}
	\caption{\label{fig:SysBlockDiagram}Control system block diagram}
\end{center}
\end{figure}

\subsection{Assumptions}

There are several assumptions that must be made in order to complete this analysis. My general assumptions for the parameters are summarized in Table ~\ref{tab:assumptions} below.

\begin{table}[H]
\begin{center}
\begin{tabular}{| l | l |}
	\hline
	Parameter & System \\ \hline
	\textit{K\textsubscript{pot}} & 0.318 \\ \hline
	\textit{K} & 1 \\ \hline
	\textit{K\textsubscript{l}} & 100 \\ \hline
	\textit{a} & 100 \\ \hline
	\textit{K\textsubscript{m}} & 2.083 \\ \hline
	\textit{a\textsubscript{m}} & 1.71 \\ \hline
	\textit{K\textsubscript{g}} & 0.1 \\ \hline
\end{tabular}
\caption{\label{tab:assumptions}Assumptions made for analysis of the system \cite{nise}}
\end{center}
\end{table}

For this analysis, I also make the following assumptions:
\begin{itemize}
	\item There is no disturbances or interference in the signals between subsystems.
	\item "Motor" in the system is an armature controlled DC servo motor.
\end{itemize}

\section{Analysis}

Analysis will be supplemented with MatLab to plot the relationships between system inputs, outputs, and gains.

\subsection{Potentiometer}

The input potentiometer is used by the user to input a desired azimuth angle $\theta$ for the system.

\begin{figure}[H]
\begin{center}
	\includegraphics[width=0.3\textwidth]{./img/PotentiometerBlock.png}
	\caption{\label{fig:pot}Input Potentiometer}
\end{center}
\end{figure}

From Figure ~\ref{fig:pot} above, I observed that the transfer function for the input potentiometer subsystem is the potentiometer output $V_{i}(s)$ divided by the user entered azimuth angle $\theta_{i}(s)$. Therefore:

\begin{equation}
\frac{V_{i}(s)}{\Theta_i(s)} = K_{pot}\label{eq:1}
\end{equation}

Based on the Equation ~\ref{eq:1} above, and the previously stated assumption that $K_{pot} = 0.318$, I am able to plot the transfer function for the input potentiometer. I analyzed the transfer function in the time domain using plots of the step and impulse responses. For frequency domain analysis, I used Nyquist and bode plots. These plots helped me determine gain and phase in the frequency domain. The results are shown in Figures ~\ref{fig:potstep} through ~\ref{fig:potnyquist}.

\begin{figure}[H]
\begin{center}
	\includegraphics[width=\textwidth]{./img/PotentiometerStep.png}
	\caption{\label{fig:potstep}Input Potentiometer Step Response}
\end{center}
\end{figure}

\begin{figure}[H]
\begin{center}
	\includegraphics[width=\textwidth]{./img/PotentiometerImpulse.png}
	\caption{\label{fig:potimpulse}Input Potentiometer Impulse Response}
\end{center}
\end{figure}

\begin{figure}[H]
\begin{center}
	\includegraphics[width=\textwidth]{./img/PotentiometerBode.png}
	\caption{\label{fig:potbode}Input Potentiometer Bode Plot}
\end{center}
\end{figure}

\begin{figure}[H]
\begin{center}
	\includegraphics[width=\textwidth]{./img/PotentiometerNyquist.png}
	\caption{\label{fig:potnyquist}Input Potentiometer Nyquist Plot}
\end{center}
\end{figure}

Functionally, the feedback potentiometer signal will be the same as the input potentiometer.

\subsection{Preamplifier}

The preamplifier is shown in Figure ~\ref{fig:preamp} below.

\begin{figure}[H]
\begin{center}
	\includegraphics[width=0.3\textwidth]{./img/PreamplifierBlock.png}
	\caption{\label{fig:preamp}Preamplifier}
\end{center}
\end{figure}

As with the potentiometer, I can determine by inspection that the transfer function of the preamplifier is equal to the output divided by the input. Therefore:

\begin{equation}
\frac{V_{p}(s)}{V_{e}(s)} = K\label{eq:2}
\end{equation}

Using my assumptions above that $K = 1$, once again I plotted the step and impulse responses, as well as the Nyquist and bode plots. The results of this analysis are shown in Figures ~\ref{fig:preampstep} through ~\ref{fig:preampnyquist}.

\begin{figure}[H]
\begin{center}
	\includegraphics[width=\textwidth]{./img/PreampStep.png}
	\caption{\label{fig:preampstep}Preamplifier Step Response}
\end{center}
\end{figure}

\begin{figure}[H]
\begin{center}
	\includegraphics[width=\textwidth]{./img/PreampImpulse.png}
	\caption{\label{fig:preampimpulse}Preamplifier Impulse Response}
\end{center}
\end{figure}

\begin{figure}[H]
\begin{center}
	\includegraphics[width=\textwidth]{./img/PreampBode.png}
	\caption{\label{fig:preampbode}Preamplifier Bode Plut}
\end{center}
\end{figure}

\begin{figure}[H]
\begin{center}
	\includegraphics[width=\textwidth]{./img/PreampNyquist.png}
	\caption{\label{fig:preampnyquist}Preamplifier Nyquist Plot}
\end{center}
\end{figure}

\subsection{Power Amplifier}

The power amplifier subsystem is shown in Figure ~\ref{fig:amp} below.

\begin{figure}[H]
\begin{center}
	\includegraphics[width=0.3\textwidth]{./img/PowerAmplifierBlock.png}
	\caption{\label{fig:amp}Power Amplifier}
\end{center}
\end{figure}

Once again using inspection, I found the transfer function for the amplifier to be:

\begin{equation}
\frac{E_{a}(s)}{V_{p}(s)} = \frac{K_{1}}{s+a}\label{eq:3}
\end{equation}

Based on assumptions above, $K_{a} = a = 100$. Once again I will use the step and impulse responses, bode and Nyquist plots. The results of this analysis are shown below in Figures ~\ref{fig:ampstep} through ~\ref{fig:ampnyquist}.

\begin{figure}[H]
\begin{center}
	\includegraphics[width=\textwidth]{./img/AmpStep.png}
	\caption{\label{fig:ampstep}Power Amplifier Step Response}
\end{center}
\end{figure}

\begin{figure}[H]
\begin{center}
	\includegraphics[width=\textwidth]{./img/AmpImpulse.png}
	\caption{\label{fig:ampimpulse}Power Amplifier Impulse Response}
\end{center}
\end{figure}

\begin{figure}[H]
\begin{center}
	\includegraphics[width=\textwidth]{./img/AmpBode.png}
	\caption{\label{fig:ampbode}Power Amplifier Bode Plot}
\end{center}
\end{figure}

\begin{figure}[H]
\begin{center}
	\includegraphics[width=\textwidth]{./img/AmpNyquist.png}
	\caption{\label{fig:ampnyquist}Power Amplifier Nyquist Plot}
\end{center}
\end{figure}

\subsection{Motor and Load}

The motor and load subsystem is shown in Figure ~\ref{fig:motor} below.

\begin{figure}[H]
\begin{center}
	\includegraphics[width=0.3\textwidth]{./img/MotorBlock.png}
	\caption{\label{fig:motor}Motor and Load}
\end{center}
\end{figure}

By inspection, I found the transfer function of this subsystem to be:

\begin{equation}
\frac{\theta_{m}(s)}{E_{a}(s)} = \frac{K_{m}}{s(s+a_{m})}\label{eq:4}
\end{equation}

In Equation ~\ref{eq:4} above, $K_{m}$ is assumed to equal 2.083, and $a_{m}$ is assumed to be 1.71. Using the equation and the previously stated assumptions, I created step and impulse response plots, and well as bode and Nyquist plots. The results of this analysis are shown in Figures ~\ref{fig:motorstep} through ~\ref{fig:motornyquist} below.

\begin{figure}[H]
\begin{center}
	\includegraphics[width=\textwidth]{./img/MotorStep.png}
	\caption{\label{fig:motorstep}Motor and Load Step Response}
\end{center}
\end{figure}

\begin{figure}[H]
\begin{center}
	\includegraphics[width=\textwidth]{./img/MotorImpulse.png}
	\caption{\label{fig:motorimpulse}Motor and Load Impulse Response}
\end{center}
\end{figure}

\begin{figure}[H]
\begin{center}
	\includegraphics[width=\textwidth]{./img/MotorBode.png}
	\caption{\label{fig:motorbode}Motor and Load Bode Plot}
\end{center}
\end{figure}

\begin{figure}[H]
\begin{center}
	\includegraphics[width=\textwidth]{./img/MotorNyquist.png}
	\caption{\label{fig:motornyquist}Motor and Load Nyquist Plot}
\end{center}
\end{figure}

\subsection{Gears}

The block diagram of the gears subsystem is found in Figure ~\ref{fig:gears} below. Note that that feedback for the system also is part of the output of this subsystem.

\begin{figure}[H]
\begin{center}
	\includegraphics[width=0.3\textwidth]{./img/GearsBlock.png}
	\caption{\label{fig:gears}Gears}
\end{center}
\end{figure}

Using inspection, the transfer function for this subsystem is as follows:

\begin{equation}
\frac{\theta_{o}(s)}{\theta_{m}(s)} = K_{g}\label{eq:5}
\end{equation}

In Equation ~\ref{eq:5} above, I assume that $K_{g} = 0.1$. I again found the step and impulse responses, and bode and Nyquist plots. The plots from this analysis are shown in Figures ~\ref{fig:gearsstep} through ~\ref{fig:gearsnyquist}.

\begin{figure}[H]
\begin{center}
	\includegraphics[width=\textwidth]{./img/GearsStep.png}
	\caption{\label{fig:gearsstep}Gears Step Response}
\end{center}
\end{figure}

\begin{figure}[H]
\begin{center}
	\includegraphics[width=\textwidth]{./img/GearsImpulse.png}
	\caption{\label{fig:gearsimpulse}Gears Impulse Response}
\end{center}
\end{figure}

\begin{figure}[H]
\begin{center}
	\includegraphics[width=\textwidth]{./img/GearsBode.png}
	\caption{\label{fig:gearsbode}Gears Bode Plot}
\end{center}
\end{figure}

\begin{figure}[H]
\begin{center}
	\includegraphics[width=\textwidth]{./img/GearsNyquist.png}
	\caption{\label{fig:gearsnyquist}Gears Nyquist Plot}
\end{center}
\end{figure}

\subsection{Complete System}

Based on my analysis of the plots in Sections ~\ref{subsec:Potentiometer} through ~\ref{subsec:Gears} above, I can see that the transfer functions of the potentiometers, preamplifier, and gears all remain constant over time. This is all based on the fact that my analysis of the system takes place under a limited set of conditions, mainly $K_{pot}$. If $K_{pot}$ changed, so too would the associated transfer functions.

To determine the overall transfer function of the system, I reduced the block diagram given in Figure ~\ref{fig:SysBlockDiagram} using the process shown in Figure ~\ref{fig:reduction}.

\begin{figure}[H]
\begin{center}
	\includegraphics[width=\textwidth]{./img/BlockDiagramReduction.png}
	\caption{\label{fig:reduction}Process of block diagram reduction}
\end{center}
\end{figure}

By utilizing my assumptions and the reduced block diagram, I can deduce the overall transfer function of the system to be:

\begin{equation}
\frac{\theta_{o}(s)}{\theta_{i}(s)} = \frac{6.624}{s^{3}+101.71s^{2}+171s+6.624}\label{eq:6}
\end{equation}

I plotted this transfer function using the same four plots as each subcomponent. Those plots are shown in Figures ~\ref{fig:completestep} through \ref{fig:completenyquist} below.

\begin{figure}[H]
\begin{center}
	\includegraphics[width=\textwidth]{./img/CompleteStep.png}
	\caption{\label{fig:completestep}Complete Step Response}
\end{center}
\end{figure}

\begin{figure}[H]
\begin{center}
	\includegraphics[width=\textwidth]{./img/CompleteImpulse.png}
	\caption{\label{fig:completeimpulse}Complete Impulse Response}
\end{center}
\end{figure}

\begin{figure}[H]
\begin{center}
	\includegraphics[width=\textwidth]{./img/CompleteBode.png}
	\caption{\label{fig:completebode}Complete Bode Plot}
\end{center}
\end{figure}

\begin{figure}[H]
\begin{center}
	\includegraphics[width=\textwidth]{./img/CompleteNyquist.png}
	\caption{\label{fig:completenyquist}Complete Nyquist Plot}
\end{center}
\end{figure}

Without feedback, the step response would be as shown in Figure ~\ref{fig:completewo}.

\begin{figure}[H]
\begin{center}
	\includegraphics[width=\textwidth]{./img/CompleteStepNoFeedback.png}
	\caption{\label{fig:completewo}Complete System Step Reponse without Feedback Potentiometer}
\end{center}
\end{figure}

\section{Conclusion}

In conclusion, the feedback potentiometer will provide critical input to prevent over or undershoot of the antenna azimuth as it approaches the azimuth input by the user. As stated in the introduction, it is critical that the antenna be aligned accurately and quickly. Given what I have learned about the response characteristics of this system, I recommend that the system should slightly increase the integral gain to ensure that the system more accurately approach the desired azimuth other time.

\pagebreak

\textbf{Appendices}

\begin{appendices}

\section{MatLab Gain and Plot Script}

\lstinputlisting[language=Octave]{analysis.m}

\end{appendices}

\begin{thebibliography}{9}

\bibitem{nise}
  Norman S. Nise,
  Control Systems Engineering,
  John Wiley and Sons, New York, New York,
  7th edition,
  2015.

\end{thebibliography}
\end{document}